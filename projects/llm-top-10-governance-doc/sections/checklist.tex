% !TEX root = owasp-doc.tex

% ================================================
%	LLM Strategy
% ================================================

\headerimage
\chapter{Checklist}

\section{Adversarial Risk}
Adversarial Risk includes competitors and attackers.

\begin{minipage}{\linewidth}
\begin{checklist}
  \item Scrutinize how competitors are investing in artificial intelligence. Although there are risks in AI adoption, there are also business benefits that may impact future market positions.
  \item Investigate the impact of current controls, such as password resets, which use voice recognition which may no longer provide the appropriate defensive security from new GenAI enhanced attacks.
  \item Update the Incident Response Plan and playbooks for GenAI enhanced attacks and AIML specific incidents.
\end{checklist}
\end{minipage}

\section{Threat Modeling}
Threat modeling is highly recommended to identify threats and examine processes and security defenses. Threat modeling is a set of systematic, repeatable processes that enable making reasonable security decisions for applications, software, and systems. Threat modeling for GenAI accelerated attacks and before deploying LLMs is the most cost effective way to Identify and mitigate risks, protect data, protect privacy, and ensure a secure, compliant integration within the business.

\begin{minipage}{\linewidth}
\begin{checklist}
  \item How will attackers accelerate exploit attacks against the organization, employees, executives, or users? Organizations should anticipate "hyper-personalized" attacks at scale using Generative AI. LLM-assisted Spear Phishing attacks are now exponentially more effective, targeted, and weaponized for an attack.
  \item How could GenAI be used for attacks on the business's customers or clients through spoofing or GenAI generated content?
  \item Can the business detect and neutralize harmful or malicious inputs or queries to LLM solutions?
  \item Can the business safeguard connections with existing systems and databases with secure integrations at all LLM trust boundaries?
  \item Does the business have insider threat mitigation to prevent misuse by authorized users?
  \item Can the business prevent unauthorized access to proprietary models or data to protect Intellectual Property?
  \item Can the business prevent the generation of harmful or inappropriate content with automated content filtering?
\end{checklist}
\end{minipage}

\section{AI Asset Inventory}
An AI asset inventory should apply to both internally developed and external or third-party solutions.

\begin{minipage}{\linewidth}
\begin{checklist}
  \item Catalog existing AI services, tools, and owners. Designate a tag in asset management for specific inventory.
  \item Include AI components in the Software Bill of Material (SBOM), a comprehensive list of all the software components, dependencies, and metadata associated with applications.
  \item Catalog AI data sources and the sensitivity of the data (protected, confidential, public)
  \item Establish if pen testing or red teaming of deployed AI solutions is required to determine the current attack surface risk.
  \item Create an AI solution onboarding process.
  \item Ensure skilled IT admin staff is available either internally or externally, following SBoM requirements.
\end{checklist}
\end{minipage}

\section{AI Security and Privacy Training}

\begin{minipage}{\linewidth}
\begin{checklist}
  \item Actively engage with employees to understand and address concerns with planned LLM initiatives.
  \item Establish a culture of open, and transparent communication on the organization's use of predictive or generative AI within the organization process, systems, employee management and support, and customer engagements and how its use is governed, managed, and risks addressed.
  \item Train all users on ethics, responsibility, and legal issues such as warranty, license, and copyright.
  \item Update security awareness training to include GenAI related threats. Voice cloning and image cloning, as well as in anticipation of increased spear phishing attacks
  \item Any adopted GenAI solutions should include training for both DevOps and cybersecurity for the deployment pipeline to ensure AI safety and security assurances.
\end{checklist}
\end{minipage}

\section{Establish Business Cases}
Solid business cases are essential to determining the business value of any proposed AI solution, balancing risk and benefits, and evaluating and testing return on investment. There are an enormous number of potential use cases; a few examples are provided.

\begin{minipage}{\linewidth}
\begin{checklist}
  \item Enhance customer experience
  \item Better operational efficiency
  \item Better knowledge management
  \item Enhanced innovation
  \item Market Research and Competitor Analysis
  \item Document creation, translation, summarization, and analysis
\end{checklist}
\end{minipage}

\clearpage

\section{Governance}
Corporate governance in LLM is needed to provide organizations with transparency and accountability. Identifying AI platform or process owners who are potentially familiar with the technology or the selected use cases for the business is not only advised but also necessary to ensure adequate reaction speed that prevents collateral damages to well established enterprise digital processes.

\begin{minipage}{\linewidth}
\begin{checklist}
  \item Establish the organization\'s AI RACI chart (who is responsible, who is accountable, who should be consulted, and who should be informed)
  \item Document and assign AI risk, risk assessments, and governance responsibility within the organization.
  \item Establish data management policies, including technical enforcement, regarding data classification and usage limitations. Models should only leverage data classified for the minimum access level of any user of the system. For example, update the data protection policy to emphasize not to input protected or confidential data into nonbusiness-managed tools.
  \item Create an AI Policy supported by established policy (e.g., standard of good conduct, data protection, software use)
  \item Publish an acceptable use matrix for various generative AI tools for employees to use.
  \item Document the sources and management of any data that the organization uses from the generative LLM models.
\end{checklist}
\end{minipage}

\clearpage

\section{Legal}
Many of the legal implications of AI are undefined and potentially very costly. An IT, security, and legal partnership is critical to identifying gaps and addressing obscure decisions.

\begin{minipage}{\linewidth}
\begin{checklist}
  \item Confirm product warranties are clear in the product development stream to assign who is responsible for product warranties with AI.
  \item Review and update existing terms and conditions for any GenAI considerations.
  \item Review AI EULA agreements. End-user license agreements for GenAI platforms are very different in how they handle user prompts, output rights and ownership, data privacy, compliance, liability, privacy, and limits on how output can be used.
  \item Organizations EULA for customers, Modify end-user agreements to prevent the organization from incurring liabilities related to plagiarism, bias propagation, or intellectual property infringement through AI-generated content.
  \item Review existing AI-assisted tools used for code development. A chatbot's ability to write code can threaten a company's ownership rights to its product if a chatbot is used to generate code for the product. For example, it could call into question the status and protection of the generated content and who holds the right to use the generated content.
  \item Review any risks to intellectual property. Intellectual property generated by a chatbot could be in jeopardy if improperly obtained data was used during the generative process, which is subject to copyright, trademark, or patent protection. If AI products use infringing material, it creates a risk for the outputs of the AI, which may result in intellectual property infringement.
  \item Review any contracts with indemnification provisions. Indemnification clauses try to put the responsibility for an event that leads to liability on the person who was more at fault for it or who had the best chance of stopping it. Establish guardrails to determine whether the provider of the AI or its user caused the event, giving rise to liability.
  \item Review liability for potential injury and property damage caused by AI systems.
  \item Review insurance coverage. Traditional (D\&O) liability and commercial general liability insurance policies are likely insufficient to fully protect AI use.
  \item Identify any copyright issues. Human authorship is required for copyright. An organization may also be liable for plagiarism, propagation of bias, or intellectual property infringement if LLM tools are misused.
  \item Ensure agreements are in place for contractors and appropriate use of AI for any development or provided services.
  \item Restrict or prohibit the use of generative AI tools for employees or contractors where enforceable rights may be an issue or where there are IP infringement concerns.
  \item Assess and AI solutions used for employee management or hiring could result in disparate treatment claims or disparate impact claims.
  \item Make sure the AI solutions do not collect or share sensitive information without proper consent or authorization.
\end{checklist}
\end{minipage}

\clearpage

\section{Regulatory}
The EU AI Act is anticipated to be the first comprehensive AI law but will apply in 2025 at the earliest. The EU's General Data Protection Regulation (GDPR) does not specifically address AI but includes rules for data collection, data security, fairness and transparency, accuracy and reliability, and accountability, which can impact GenAI use. In the United States, AI regulation is included within broader consumer privacy laws. Ten US states have passed laws or have laws that will go into effect by the end of 2023.
Canada has so far only published a Voluntary Code of Conduct on the Responsible Development and Management of Advanced Generative AI Systems, however, the Artificial Intelligence and Data Act (AIDA) will have stronger requirements.
Federal organizations such as the US Equal Employment Opportunity Commission (EEOC), the Consumer Financial Protection Bureau (CFPB), the Federal Trade Commission (FTC), and the US Department of Justice\'s Civil Rights Division (DOJ) are closely monitoring hiring fairness.

\begin{minipage}{\linewidth}
\begin{checklist}
  \item Determine Country, State, or other Government specific AI compliance requirements.
  \item Determine compliance requirements for restricting electronic monitoring of employees and employment-related automated decision systems (Vermont, California, Maryland, New York, New Jersey)
  \item Determine compliance requirements for consent for facial recognition and the AI video analysis required (Illinois, Maryland, Washington, Vermont)
  \item Review any AI tools in use or being considered for employee hiring or management.
  \item Confirm the vendor\'s compliance with applicable AI laws and best practices.
  \item Ask and document any products using AI during the hiring process. Ask how the model was trained, and how it is monitored, and track any corrections made to avoid discrimination and bias.
  \item Ask and document what accommodation options are included.
  \item Ask and document whether the vendor collects confidential data.
  \item Ask how the vendor or tool stores and deletes data and regulates the use of facial recognition and video analysis tools during pre-employment.
  \item Review other organization-specific regulatory requirements with AI that may raise compliance issues. The Employee Retirement Income Security Act of 1974, for instance, has fiduciary duty requirements for retirement plans that a chatbot might not be able to meet.
\end{checklist}
\end{minipage}

\section{Using or Implementing Large Language Model Solutions}

\begin{minipage}{\linewidth}
\begin{checklist}
  \item Threat Model LLM components and architecture trust boundaries.
  \item Data Security, verify how data is classified and protected based on sensitivity, including personal and proprietary business data. (How are user permissions managed, and what safeguards are in place?)
  \item Access Control, implement least privilege access controls and implement defense-in-depth measures
  \item Training Pipeline Security, require rigorous control around training data governance, pipelines, models, and algorithms.
  \item Input and Output Security, evaluate input validation methods, as well as how outputs are filtered, sanitized, and approved.
  \item Monitoring and Response, map workflows, monitoring, and responses to understand automation, logging, and auditing. Confirm audit records are secure.
  \item Include application testing, source code review, vulnerability assessments, and red teaming in the production release process.
  \item Check for existing vulnerabilities in the LLM model or supply chain.
  \item Look into the effects of threats and attacks on LLM solutions, such as prompt injection, the release of sensitive information, and process manipulation.
  \item Investigate the impact of attacks and threats to LLM models, including model poisoning, improper data handling, supply chain attacks, and model theft.
  \item Supply Chain Security, request third-party audits, penetration testing, and code reviews for third-party providers. (both initially and on an ongoing basis)
  \item Infrastructure Security, ask how often a vendor performs resilience testing? What are their SLAs in terms of availability, scalability, and performance?
  \item Update incident response playbooks and include an LLM incident in tabletop exercises.
  \item Identify or expand metrics to benchmark generative cybersecurity AI against other approaches to measure expected productivity improvements.
\end{checklist}
\end{minipage}

\section{Testing, Evaluation, Verification, and Validation \(TEVV\)}
NIST AI Framework recommends a continuous TEVV process throughout the AI lifecycle which includes the AI system operators, domain experts, AI designers, users, product developers, evaluators, and auditors. TEVV includes a range of tasks such as system validation, integration, testing, recalibration, and ongoing monitoring for periodic updates to navigate the risks and changes of the AI system.

\begin{minipage}{\linewidth}
\begin{checklist}
  \item Establish continuous testing, evaluation, verification, and validation throughout the AI model lifecycle.
  \item Provide regular executive metrics and updates on AI Model functionality, security, reliability, and robustness.
\end{checklist}
\end{minipage}

\clearpage

\section{Model Cards and Risk Cards}
Model cards and risk cards are foundational elements for increasing the transparency, accountability, and ethical deployment of Large Language Models (LLMs). Model cards help users understand and trust AI systems by providing standardized documentation on their design, capabilities, and constraints, leading them to make educated and safe applications. Risk cards supplement this by openly addressing potential negative consequences, such as biases, privacy problems, and security vulnerabilities, which encourages a proactive approach to harm prevention. These documents are critical for developers, users, regulators, and ethicists equally since they establish a collaborative atmosphere in which AI's social implications are carefully addressed and handled. These cards, developed and maintained by the organizations that created the models, play an important role in ensuring that AI technologies fulfill ethical standards and legal requirements, allowing for responsible research and deployment in the AI ecosystem.

Model cards include key attributes associated with the ML model:

\begin{itemize}
  \item \textbf{Model details}: Basic information about the model, i.e., name, version, and type ( neural network, decision tree, etc.), and the intended use case.
  \item \textbf{Model architecture}: Includes a description of the structure of the model, such as the number and type of layers, activation functions, and other key architectural choices.
  \item \textbf{Training data and methodology}: Information about the data used to train the model, such as the size of the dataset, the data sources, and any preprocessing or data augmentation techniques used. It also includes details about the training methodology, such as the optimizer used, the loss function, and any hyperparameters that were tuned.
  \item \textbf{Performance metrics}: Information about the model's performance on various metrics, such as accuracy, precision, recall, and F1 score. It may also include information about how the model performs on different subsets of the data.
  \item \textbf{Potential biases and limitations}: Lists potential biases or limitations of the model, such as imbalanced training data, overfitting, or biases in the model's predictions. It may also include information about the model's limitations, such as its ability to generalize to new data or its suitability for certain use cases.
  \item \textbf{Responsible AI considerations}:  Any ethical or responsible AI considerations related to the model, such as privacy concerns, fairness, and transparency, or potential societal impacts of the model's use. It may also include recommendations for further testing, validation, or monitoring of the model.
\end{itemize}

\clearpage

The precise features contained in a model card may differ based on the model's context and intended usage, but the purpose is to give openness and accountability in the creation and deployment of machine learning models.

\begin{minipage}{\linewidth}
\begin{checklist}
  \item Review a model\'s model card
  \item Review risk card if available
  \item Establish a process to track and maintain model cards for any deployed model including models used through a third party.
\end{checklist}
\end{minipage}

\section{RAG: Large Language Model Optimization}
Fine tuning, the traditional method for optimizing a pre-trained model, involved retraining an existing model on new, and domain-specific data, modifying it for performance on a task or application. Fine-tuning is expensive but essential to improve performance.

Retrieval-Augmented Generation \(RAG\) has evolved as a more effective way of optimizing and augmenting the capabilities of large language models by retrieving pertinent data from up to date available knowledge sources. RAG can be customized for specific domains, optimizing the retrieval of domain-specific information and tailoring the generation process to the nuances of specialized fields. RAG is seen as a more efficient and transparent method for LLM optimization, particularly for problems where labeled data is limited or expensive to collect. One of the primary advantages of RAG is its support for continuous learning since new information can be continually updated at the retrieval stage.

The RAG implementation involves several key steps starting from embedding model deployment, indexing the knowledge library, to retrieving the most relevant documents for query processing. Efficient retrieval of the relevant context is made based on vector databases which are used for storage and querying of document embeddings.

\textbf{RAG Reference}

\begin{minipage}{\linewidth}
\begin{checklist}
  \item \href{https://vitalflux.com/retrieval-augmented-generation-rag-llm-examples/}{Retrieval Augmented Generation \(RAG\) \& LLM: Examples}
  \item \href{https://towardsdatascience.com/12-rag-pain-points-and-proposed-solutions-43709939a28c}{12 RAG Pain Points and Proposed Solutions}
\end{checklist}
\end{minipage}

\section{AI Red Teaming}
AI Red Teaming is an adversarial attack test simulation of the AI System to validate there aren\'t any existing vulnerabilities which can be exploited by an attacker. It is a recommended practice by many regulatory and AI governing bodies including the Biden administration. Red-teaming alone is not a comprehensive solution to validate all real-world harms associated with AI systems and should be included with other forms of testing, evaluation, verification, and validation such as algorithmic impact assessments and external audits.

\begin{minipage}{\linewidth}
\begin{checklist}
  \item Incorporate Red Team testing as a standard practice for AI Models and
  applications.
\end{checklist}
\end{minipage}